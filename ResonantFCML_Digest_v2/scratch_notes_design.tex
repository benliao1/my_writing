


\newpage

\section*{Design and Component Selection}
6-level FCML 

$P_{IN}=200W$ (arbitrary).

$f_{SW}=100kHz$ (arbitrary)

$V_{HI} = 200V$ to ensure nominal blocking voltage of 40V on each FET; allows 100\% overshoot on 100V EPC2218 (3.2m$\Omega$). (Can use 0$\Omega$ $R_{GATE}$ to minimize overlap loss interference)

$V_{LO}=40V$ implies $I_{OUT}=5A$.

\vspace{1em}
Caps:

5750 (2220) C5750C0G2E154J230KN 250V  C0G 0.15uF. 

Energy: four caps in parallel $\frac{1}{2}4CV^2 = \frac{1}{2}(4\times 0.15\mu F)(200V)^2 = 0.012 \; Joules$

Volume: 0.0057 * 0.005 * 0.0045 = 0.12825e-6 $m^2$

Energy Density = 93,567 J/$m^3$

\vspace{2em}
Inductor: Example Coilcraft MSS1260H

12.3mm x 12.3mm x 6mm = 0.90774e-6 $m^3$

$1\mu H$ at 20A max

Energy = $\frac{1}{2} L I^2 = \frac{1}{2} (1\mu H) (20A)^2$ = 0.0002 Joules

energy Density = 220 J/$m^3$

optimization algorithm currently gives: C0 = 133$\mu$F, and L = 4.47nH.  $\longleftarrow$ Am not happy with this and want to revisit code. Could be correct, but this compounds problem of poor Q-factor and loss (oh boy...) Hmm na, think there's error: this would give t1=2.4$\mu s$ $q_H=11.9\mu$C, or $\Delta V=0.09V$. Seems ridiculuously small ripple???

\vspace{1em}
Am planning on just running with 2$\times$0.3$\mu$F and 1$\mu$H parts. 

This gives $T_{SW,nom} = 2 \times \pi \sqrt{(0.6\mu F)(1\mu H)} + 3 \times  \pi \sqrt{(0.3\mu F)(1\mu H)} = 10\mu s \quad$ or $\quad f_{SW}= 100\;kHz$

We can double up caps (plenty of pad space) or increase L if we want to lower the resonant freq $f_{nom}$; This will allow us to push Gamma lower without seeing annoying high-freq effects.

This component selection will ensure high Q-factor for clear resonant operation.

$Q = \frac{1}{R}\sqrt{\frac{L}{C}} = \frac{1}{\sim 15m\Omega}\sqrt{\frac{1\mu H}{0.6\mu F}} \quad = \quad \sim 86 $





